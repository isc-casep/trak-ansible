\documentclass[12pt,landscape]{beamer}
\usepackage[utf8]{inputenc}
\usepackage[english]{babel}
\usepackage{amsmath}
\usepackage{enumitem}
\usepackage{amsfonts}
\usepackage{amssymb}
\usetheme{Amsterdam}
\usepackage{listings}
\usepackage{url}
\usepackage{enumitem}
\usepackage{graphicx}

\usepackage[nodayofweek]{datetime}

\setbeamertemplate{caption}[numbered]
\graphicspath{{images/}}
\newcounter{MyCounter}

\expandafter\def\expandafter\insertshorttitle\expandafter{%
  \insertshorttitle\hfill%
  \insertframenumber\,/\,\inserttotalframenumber}

\setlist[itemize,1]{label=$\times$}
\setlist[itemize,2]{label=$\checkmark$}
\setlist[itemize,3]{label=$\diamond$}
\setlist[itemize,4]{label=$\bullet$}

\definecolor{dkgreen}{rgb}{0,0.6,0}
\definecolor{gray}{rgb}{0.5,0.5,0.5}
\definecolor{mauve}{rgb}{0.58,0,0.82}
\definecolor{lightgray}{gray}{0.9}

\lstset{ %
  basicstyle=\tiny,           % the size of the fonts that are used for the code
  numbers=left,                   % where to put the line-numbers
  numberstyle=\tiny\color{gray},  % the style that is used for the line-numbers
  stepnumber=1,                   % the step between two line-numbers. If it's 1, each line
                                  % will be numbered
  numbersep=4pt,                  % how far the line-numbers are from the code
  backgroundcolor=\color{white},      % choose the background color. You must add \usepackage{color}
  showspaces=false,               % show spaces adding particular underscores
  showstringspaces=false,         % underline spaces within strings
  showtabs=false,                 % show tabs within strings adding particular underscores
  frame=single,                   % adds a frame around the code
  rulecolor=\color{black},        % if not set, the frame-color may be changed on line-breaks within not-black text (e.g. commens (green here))
  tabsize=2,                      % sets default tabsize to 2 spaces
  captionpos=b,                   % sets the caption-position to bottom
  breaklines=true,                % sets automatic line breaking
  breakatwhitespace=false,        % sets if automatic breaks should only happen at whitespace
  title=\lstname,                   % show the filename of files included with \lstinputlisting;
                                  % also try caption instead of title
  keywordstyle=\color{blue},          % keyword style
  commentstyle=\color{dkgreen},       % comment style
  stringstyle=\color{mauve},         % string literal style
  escapeinside={\%*}{*)},            % if you want to add LaTeX within your code
  morekeywords={*,...}               % if you want to add more keywords to the set
}

\author{Carlos ``casep'' Sepulveda \\ casep@intersystems.com}
\date{\today}
\title{TrakCare deployment using Ansible}
\subtitle{Go away or I will replace you with a simple yaml script}

\begin{document}

\begin{frame}
  \titlepage
\end{frame}

\begin{frame}
  \frametitle{Plan}
  \tableofcontents
\end{frame}

\section{Introduction}
\begin{frame}[allowframebreaks]
  \frametitle{Motivation}

\lstinputlisting[language=bash,caption={Extract of the old code.}]{scripts/oldcode.txt}

Just impossible to keep the old code properly updated. A mess of different perl, bash, expect and similar scripts.

\end{frame}

\section{Background}

\begin{frame}
 \textbf{Background}
\end{frame}

\begin{frame}
  \frametitle{Ansible?}
  
  ``Ansible is a radically simple IT automation engine that automates cloud provisioning, configuration management, application deployment, intra-service orchestration, and many other IT needs''\cite{ansible2}

  ``...the simple, yet powerful IT automation engine that thousands of companies are using to drive complexity out of their environments and accelerate DevOps initiatives''\cite{ansible}
          
\end{frame}

\begin{frame}
  \frametitle{Why Ansible?}
  
  ``It uses no agents and no additional custom security infrastructure, so it's easy to deploy - and most importantly, it uses a very simple language (YAML, in the form of Ansible Playbooks) that allow you to describe your automation jobs in a way that approaches plain English''\cite{ansible2}
          
\end{frame}

\begin{frame}
  \frametitle{Simple sintax}

  \lstinputlisting[language=bash,caption={Example of sintax for a task.}]
{scripts/helloworld.yml}

\end{frame}        

\begin{frame}
  \frametitle{Simple to run}

  \lstinputlisting[language=bash,caption={Example of playbook execution.}]
{scripts/ansible_run.txt}

\end{frame}

\begin{frame}
  \frametitle{Simple to debug (maybe)}

  \lstinputlisting[language=bash,caption={Example of failed execution.}]
{scripts/error.txt}

\end{frame}

\begin{frame}
  \frametitle{Want to know more?}

Intro to Playbooks\cite{intro_ansible}.

\end{frame}

\section{The tool}

\begin{frame}
  \frametitle{Deploying TrakCare}

  \lstinputlisting[language=bash,caption={TrakCare up and running!}]
{scripts/trakcare_run.txt}

\end{frame}

\begin{frame}[allowframebreaks]
  \frametitle{Components of the the project}

  \lstinputlisting[language=bash,caption={Files in the project.}]{scripts/tree.txt}

\end{frame}

\begin{frame}
  \frametitle{ansible.cfg}

Holds the definition for your \textbf{inventory}. Ansible works against multiple managed nodes or ``hosts'' in your infrastructure at the same time, using a list or group of lists know as inventory\cite{inventory_ansible}.

Also stored here, the definition for the \textbf{remote\_user}. This user is the one used to access the remote servers during deployment.

\end{frame}

\begin{frame}[allowframebreaks]
  \frametitle{group\_vars}

While automation exists to make it easier to make things repeatable, all systems are not exactly alike; some may require configuration that is slightly different from others. In some instances, the observed behavior or state of one system might influence how you configure other systems. 

For example, you might need to find out the IP address of a system and use it as a configuration value on another system.

Ansible uses variables to help deal with differences between systems\cite{variables_ansible}.

A variable is defined in the format: 

name\_of\_the\_variable: value

And it is referenced in the script as:

{{ name\_of\_the\_variable }}
\bigskip

Example of variables:
  \lstinputlisting[language=bash,caption={Variables for a TrakCare 2019 deployment.}]
{scripts/variables.txt}

\end{frame}

\begin{frame}
  \frametitle{hosts}

Here you can find the definition for the groups of servers. 

Each group will have different tasks associated to it.

\lstinputlisting[language=bash,caption={Example definition for hosts and groups.}]{scripts/hosts.txt}

More information\cite{inventory_ansible}. 

\end{frame}

\begin{frame}
  \frametitle{site.yml}

This file holds the main configuration for the site.

It includes, which roles to load and which hosts to manage.

More definition on importing playbooks\cite{import_playbook}.

\end{frame}

\begin{frame}
  \frametitle{roles}

Roles are ways of automatically loading certain vars\_files, tasks, and handlers based on a known file structure. 
Grouping content by roles also allows easy sharing of roles with other users.\cite{import_playbook}.

\lstinputlisting[language=bash,caption={Role Directory Structure.}]{scripts/roles.txt}

\end{frame}

\begin{frame}
  \frametitle{TrakCare 2019}

TrakCare 2019 configuration:

\lstinputlisting[language=bash,caption={All the changes required for a 2019 deployment.}]{scripts/2019.txt}

\end{frame}

\begin{frame}
  \frametitle{TrakCare 2020}

TrakCare 2020 configuration:

\lstinputlisting[language=bash,caption={All the changes required for a 2020 deployment.}]{scripts/2020.txt}

\end{frame}


\section{References}

\begin{frame}
 \textbf{References}
\end{frame}

\begin{frame}[allowframebreaks]
 \frametitle{References}
 \bibliography{bib}
 \bibliographystyle{plain}
\end{frame}

\begin{frame}
  \tiny
  Built using \LaTeX\ with \ $100\%$ recycled electrons
\end{frame}

\end{document}
